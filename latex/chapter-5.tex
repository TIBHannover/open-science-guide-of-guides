\documentclass{article}

\usepackage{caption}
                
\usepackage[backend=biber,hyperref=false,citestyle=authoryear,bibstyle=authoryear]{biblatex}
                
\bibliography{bibliography}
            
\usepackage{graphicx}
                
\usepackage{calc}
                
\newlength{\imgwidth}
                
\newcommand\scaledgraphics[2]{%
                
\settowidth{\imgwidth}{\includegraphics{#1}}%
                
\setlength{\imgwidth}{\minof{\imgwidth}{#2\textwidth}}%
                
\includegraphics[width=\imgwidth,height=\textheight,keepaspectratio]{#1}%
                
}
            
\begin{document}

\title{OSG005 - Open Science and Open Access Publishing }

\maketitle





\textbf{Authors}:  Maria Sael, Sabrina Gaab, Mohammad Al Nasouh, Edith Reschner


\textbf{Tags / topics} (4): Open Access, Open Science, Open Access Publishing, Open Licence, Scholarly publishing,  APCs, author rights, copyright.


\subsection{1. Guide }\label{H7151279}


\begin{itemize}
\item \textbf{Cover image:} 


\end{itemize}
\begin{figure}
\scaledgraphics{add6dfd7-ef0a-49dc-b0a6-7ae1e4e54339.png}{1}
\caption*{Screeshot from guide on RMIT website}\label{F87257141}
\end{figure}




\begin{itemize}
\item \textbf{Guide name:} "Open access Publishing'


\item \textbf{Guide citation insert from Zotero} :  \autocite{macvean_all_2021} 


\item \textbf{Type of guide parts} : step-by-step, instructions, and checklists 


\item \textbf{Summary:} 


\end{itemize}

A guide was written by Karen Macvean and published in the online library RMIT - Global University of Technology, Design and Economics to explain everything about Open Access briefly using different exploration methods such as text, explanatory videos, charts, and illustrations. The guide explains the idea behind Open Access, its models such as Gold, Hybrid, and Green Open Access. An illustration also shows the benefits of open access in different disciplines. The difference in the citation volume of Open Access publications compared to non-Open Access publications is also shown in a diagram. Further tips on how to make research more open are listed as well as information on what preprints are, why, and how preprints can be shared are listed.  The guide includes a list of open-access resources, such as Organizations, Directories, and Tools.  The guide addresses FAIR principles, policies, and ethics, data planning, storing, and sharing data.  Reading this guide will help with choosing the right type of publication, be it in journal articles, books and book chapters, conference papers, or non-traditional research (NTROs). The guide also provides an overview of copyrights and Information on Article Processing Charges (APCs) that should be checked before paying a journal.


\subsection{3. Guide}\label{H9740541}



\begin{center}
\begin{figure}
\scaledgraphics{499c4719-8346-4a67-8a5b-fcacfb0ecde0.png}{0.5}
\caption*{Abbildung 1: Cover: "Von Open Access zu Open Science"}\label{F14991031}
\end{figure}


\end{center}





\textbf{Guide name:} "Von Open Access zu Open Science: Zum Wandel digitaler Kulturen der wissenschaftlichen Kommunikation"


\textbf{Guide citation insert from Zotero:}  \autocite{heise_von_2018}


\textbf{Type of Guide: }Mit Hilfe eines Experiments werden in diesem Handbuch Chancen und Hindernisse von Open Access dargestellt. 


\textbf{Summary (Main Topics): }Mit der Digitalisierung geht der Ruf nach freiem Zugang zu wissenschaftlichen Forschungsergebnissen und einer Öffnung des Forschungsprozesses einher. Open Access und Open Science sind die Leitbegriffe dieses Transformationsprozesses, der von den einen euphorisch begrüßt und von den anderen heftig abgelehnt wird. Auf der Grundlage einer quantitativen Erhebung und eines reflexiven Experiments gibt das Buch Einblick in die aktuellen Debatten über die Chancen aber auch Hindernisse der Öffnung der Wissenschaften.





\subsection{4. Guide}\label{H1144211}


\begin{figure}
\scaledgraphics{37cc2dfd-e350-4e29-acc3-9c7f815eb133.png}{0.5}
\caption*{Cover}\label{F36070241}
\end{figure}




















\textbf{Guide name:} 

Understanding Open Access. When, why, \& how to make your work openly accessible


\textbf{Guide citation insert from Zotero:} 

\autocite{rubow_understanding_2015}


\textbf{Type of Guide:}

The basic structure of this step-by-step guide traces the process of how an author would decide whether and how to make a work openly accessible. Therefore, this design is intended to help with each step of the decision-making-process when thinking about Open Access Publishing. The aim is to provide real-life strategies and tools that authors can use to work with publishers, institutions, and funders to make their works available on the terms most consistent with their dissemination goals.


On another note, this guide is the product of extensive interviews with authors, publishers, and institutional representatives who shared their perspectives on open access options in today’s publishing environment. The information, strategies, and examples included in this guide reflect the collective wisdom of these interviewees.


\textbf{Target Group:}

The guide is for authors of all backgrounds, fields, and disciplines, from the sciences to the humanities.


\textbf{Summary:}

This Guide "Understanding Open Access" provides a scholarly author-oriented look at the ins and outs of open access publishing. The guide addresses common concerns about what "open access" means, how institutional open access requirements work, and why authors might consider making their work openly accessible online. 


This guide will help to determine whether open access is right for the interested party and their work and, if so, how to make it openly accessible. This primer on open access explains what “open access” means, addresses common concerns and misconceptions you may have about open access, and provides you with practical steps to take if you wish to make your work openly accessible.


Following the Introduction, there are three more sections at hand: Section II helps to evaluate whether to make the work openly accessible. When the decision is made, to make the work openly accessible, the reader can go on to the next section. Section III then explains how to do so by giving advices on how "open" to make the work at hand, where to make it openly availabe to the public and also how to secure the right to use third-party content in the later openly accessible work. Also included are strategies on how to make the work openly accesible while also publishing it through a conventional publisher.  Finally, the guide concludes with Section IV, a window on the future of open access.





\printbibliography[title={Literaturverzeichnis}]
\end{document}
